%%%%%%%%%%%%%%%%%%%%%%%%%%%%%%%%%%%%%%%%%%%%%%%%%%%%%%%%%%%%%%
% Modified  :   
% Created   :   
% File      :  
%%%%%%%%%%%%%%%%%%%%%%%%%%%%%%%%%%%%%%%%%%%%%%%%%%%%%%%%%%%%%%
\documentclass[11pt]{article}
\usepackage{amsfonts}
\usepackage{verbatim}
\usepackage{chicago}
%\usepackage{natbib,natbibspacing}
\usepackage[active]{srcltx}
%\oddsidemargin 0.0cm
%\evensidemargin 0.0cm
\topmargin -1.5cm
\oddsidemargin +0.3cm
\textwidth162mm
\textheight235mm
%\renewcommand{\baselinestretch}{.94}
\renewcommand\refname{\centerline{\Large\bf References}}
\renewcommand{\theequation}{\thesection.\arabic{equation}}
\makeatletter
\@addtoreset{equation}{section}
%\renewcommand{\@biblabel}[1]{#1.\hfill}
\makeatother
\begin{document}
\font\eufm = eufm10 scaled 1200
%\def\u{{\bf u}}
%\def\Beta{{\cal B}}
%\def\D{\mbox{\bf D}}
%\def\E{{\bf E}}
%\def\Ff{\mbox{\eufm F}}
%\def\N{{\cal N}}
%\def\Nb{\mathbb N}
%\def\P{{\bf P}}
%\def\Rb{\mathbb R}
%\def\vv{{\bf v}}
%\def\X{{\bf X}}
%\def\x{{\bf x}}
%\def\di{\displaystyle}
%\def\indi{1\hspace{-1,1mm}{\rm I}}
%\def\toi#1{\mbox{\kern+5pt \hbox to 25pt{\rightarrowfill}
% \kern-28pt \raise-5pt   \hbox{$\scriptstyle #1 \to \infty $ }\kern+1pt }}
%\def\too#1{\mbox{\kern+3pt \hbox to 23pt{\rightarrowfill}
% \kern-23pt \raise-5pt   \hbox{$\scriptstyle #1 \to 0 $} \kern+0pt }}
%\def\tow{\stackrel{W}{\rightarrow}}
%\def\top{{\;\buildrel \bf P  \over{\toi{n}}\;}}
%\def\tr{{\rm \bf tr}}
%\def\Var{\mbox{\bf Var}}

\def\indi{{\mathbb I}}
\def\var{\mbox{\bf var}}
\def\cov{\mbox{\bf cov}}
\def\diag{\mbox{diag}}
\def\ssc{\scriptscriptstyle}

\newtheorem{theorem}{Theorem}[section]
\newtheorem{lemma}{Lemma}[section]
\newtheorem{proposition}{Proposition}[section]
\newtheorem{corollary}{Corollary}[section]

%\pagestyle{myheadings}
%********************************************************

{\centerline{\Large\bf Linkage disequilibrium.\bigskip}}

We consider first genetic markers of two alleles in diploid case. Then the allelic marker has two levels {\small "$+$"} and {\small "$-$"}, and the genotype marker has three levels  {\small "$++$"}, {\small "$+-$"} and {\small "$--$"}. 


{\bf 1: Allelic linkage disequilibrium (LDE)} is based on the alele frequencies $p_{\ssc +}$ and $p_{\ssc -}$. For any two fixed locuses  denote $p_{xy}$ is the frequency of the variant $x$ at the first locus and variant $y$ at the second one. In these notations we have the following frequency table: \vspace{.5em}\\
%\renewcommand{\arraystretch}{.7} 
\centerline{\begin{tabular}{c|c|c|c}
 Locus $1\backslash\mbox{Locus}\; 2$   & "$+$"        & "$-$"       & Total \\
\hline      
"$+$"     & $p_{\ssc  ++}$   & $p_{\ssc  +-}$   & $p_{\ssc  +\bullet}$ \\
"$-$"     & $p_{\ssc  -+}$   & $p_{\ssc --}$   & $p_{\ssc -\bullet}$ \\
\hline
Total & $p_{\ssc \bullet +}$    & $p_{\ssc \bullet -}$    & $1$ \\
\end{tabular}}
Introduce the coefficient $D=p_{\ssc  ++}-p_{\ssc \bullet +} p_{\ssc +\bullet}$, which is the covariance between the two random variables taking values $0$ under $+$ variant and $1$ under $-$ variant (or vice versa).

There are two ways to determine the allelic LDE coefficients:
$$
r=\frac{D}{\sqrt{p_{\ssc +\bullet}(1-p_{\ssc +\bullet})p_{\ssc \bullet +}(1-p_{\ssc \bullet +)}}}
$$ 
and $r^2$ is commonly used, and
$$
D'=|D|/D_{\max},
$$
where 
$$
D_{\max}=
\cases{
\min(p_{\ssc +\bullet}p_{\ssc \bullet +},p_{\ssc -\bullet}p_{\bullet -}), \;\mbox{under}\; D<0 \cr
\min(p_{\ssc +\bullet}p_{\ssc \bullet -},p_{\ssc -\bullet}p_{\ssc \bullet +}), \;\mbox{under}\; D>0. \cr
}
$$
We are using the signed version of the $D'$ coefficient 
$$
D'_{s}=D/D_{\max}
$$
(or the $r$ coefficient) to measure allelic LDE in two locuses. 

{\bf 2: Composite LDE.} In the diploid organisms all the genome variants in two locuses (haplotypes) are collected in the following table: 
\vspace{0.5em}
\\
%\renewcommand{\arraystretch}{.7} 
\centerline{\begin{tabular}{c|c|c|c|c|c}
 Locus $1\backslash\mbox{Locus}\; 2$       & "$++$"   & "$+-$"  &  "$-+$"   & "$--$"       & Total \\
\hline      
"$++$"     & $p_{\ssc ++++}$  & $p^{\circ}_{\ssc +++-}$ & $p^{\circ}_{\ssc ++-+}$ & $p_{\ssc ++--}$   & $p_{++\bullet}$ \\
\hline 
"$+-$"     & $p^{\circ}_{\ssc +-++}$  & $p^{\circ}_{\ssc +-+-}$ & $p^{\circ}_{\ssc +--+}$ & $p^{\circ}_{\ssc +----}$   & $p_{\ssc +-\bullet}$ \\
\hline 
"$-+$"     & $p^{\circ}_{\ssc -+++}$  & $p^{\circ}_{\ssc -++-}$ & $p^{\circ}_{\ssc -+-+}$ & $p^{\circ}_{\ssc -+--}$   & $p_{\ssc -+\bullet}$ \\
\hline 
"$--$"     & $p_{\ssc --++}$  & $p^{\circ}_{\ssc --+-}$ & $p^{\circ}_{\ssc ---+}$ & $p_{\ssc ----}$   & $p_{\ssc --\bullet}$ \\
\hline
Total & $p_{\ssc \bullet ++}$    & $p_{\ssc \bullet +-}$ &  $p_{\ssc \bullet -+}$  &  $p_{\ssc \bullet --}$   & $1$ \\
\end{tabular}} 
\vspace{.5em}\\
After unification cells with identical combination we obtain 10-cells table: 
\vspace{0.5em}
\\
%\renewcommand{\arraystretch}{.7} 
\centerline{\begin{tabular}{c|c|c|c|c|c}
 Locus $1\backslash\mbox{Locus}\; 2$       & "$++$"   & "$+-$"  &  "$-+$"   & "$--$" & Total   \\
\hline      
"$++$"     & $p_{\ssc ++++}$  & \multicolumn{2}{c|}{$p_{\ssc +++-}$}  & $p_{\ssc ++--}$ & $p_{\ssc ++\bullet}$   \\
\hline  
"$+-$"     & $p_{\ssc +-++}$  & $p_{\ssc +-+-}$ & $p_{\ssc +--+}$ & $p_{\ssc +---}$ & $p_{\ssc +-\bullet}$  \\
\hline  
"$--$"     & $p_{\ssc --++}$  & \multicolumn{2}{c|}{$p_{\ssc --+-}$} & $p_{\ssc ----}$ & $p_{\ssc --\bullet}$  \\
\hline  
Total & $p_{\ssc \bullet ++}$    & \multicolumn{2}{c|}{$p_{\ssc \bullet +-}$} &  $p_{\ssc \bullet --}$   & $1$ \\
\end{tabular}}
\vspace{0.5em}
were $p_{ijks}=2p^{\circ}_{ijks}=2p^{\circ}_{ijsk}=2p^{\circ}_{jiks}$. 

The haploid (gametic)  part of the LDE: $D_g=p_{g}-p_{+\bullet}p_{\bullet +}$, where 
$$
p_{g}=p_{\ssc ++++}+p_{\ssc +++-}/2+p_{\ssc +-++}/2+(p_{\ssc +-+-}+p_{\ssc +--+})/4.
$$
The diploid (non gametic) part LDE:  $D_{d}=p_{d}-p_{\ssc +\bullet}p_{\ssc \bullet +}$ linkage disequilibrium, where 
$$
p_{d}=p_{\ssc ++++}+p_{\ssc +++-}/2+p_{\ssc +-++}/2+p_{\ssc +-+-}/2. 
$$
The composite LDE: 
$$
\Delta=D_{g}+D_{d}=p_{g}+p_{d}-2p_{\ssc +\bullet}p_{\ssc \bullet +}.
$$

If genotypes are known only the composite LDE cannot be obtained. Then one use the within chromosome equilibrium  $p_{\ssc +-+-}=p_{\ssc +--+}$. Under the within chromosome equilibrium $\Delta=2D_g$. 

The composite linkage disequilibrium correlation is defined as follows:
$$
r=\frac{\Delta}{\sqrt{(p_{+\bullet}p_{-\bullet}+D_1)(p_{\bullet +}p_{\bullet -}+D_2)}},
$$
where $D_1=\mbox{\it HWD}_1=p_{\ssc ++\bullet}-p_{\ssc +\bullet}^2$, $D_2=\mbox{\it HWD}_2=p_{\ssc \bullet ++}-p_{\ssc \bullet +}^2$. 

Under within chromosome equilibrium assumption if to set 
%\renewcommand{\arraystretch}{.7} 
\\ 
$$
\xi_1\!=\!
\cases{
-1, \; \mbox{if genotype is}\; {\scriptstyle "++"}, \cr
0, \; \mbox{if genotype is}\; {\scriptstyle "+-"}, \cr
1, \; \mbox{if genotype is}\; {\scriptstyle "--"}, \cr
}
\;
\mbox{and}
\quad
\xi_2\!=\!
\cases{
-1, \; \mbox{if genotype is}\; {\scriptstyle "++"},\cr
0, \; \mbox{if genotype is}\; {\scriptstyle "+-"}, \cr
1, \; \mbox{if genotype is}\; {\scriptstyle "--"},\cr
}
$$
then $\Delta=\cov(\xi_A,\xi_B)$ and $r=\mbox{\bf corr}(\xi_A,\xi_B)$.
\end{document}

Before starting cflculations of multiple tests the input clynical data should be checked. The phenotype should be a factor of $d$-levels. Number of the phenotype levels are calculated on the preliminary stage. Denote the common variant is {"\small $+$"} and the minor variant  is {\small "$-$"}. The genotype levels for the codominant, dominant and recessive alternatives and for the allelic approach are given in the following table. 

\centerline{
\begin{tabular}{lcccc}
Test $\backslash$ Variant &  {\small "$++$" }   &  {\small "$+-$" } &  {\small "$--$" } & Missed \\
\hline 
Codominant & $0$  & $1$ & $2$ & $3$ \\
Dominant & $0$  & $1$ & $1$ & $3$  \\ 
Recessive &  $0$   & $0$ & $1$ & $3$  \\ 
Allelic    & $(0,0)$ & $(1,0)$ & $(1,1)$ & $(3,3)$ 
\vspace{+3mm}
\end{tabular}
}
\noindent
Under the allelic approach each observaton produces two with the same phenotype and genotype belonging to $\{0,1\}$, so the population size is doubled.

The GWATCH forms $d\times 3$ contingency table $\{n_{ij}\}$ for the codominant alternative and $d\times 2$ contigency table $\{n_{ij}\}$ for other approaches, where $n_{ij}$ are the corresponding counts. Before starting multiple tests GWATCH classify the phenotype: {\it binary} or {\it multilevel}. GWATCH removes missed data for each SNP-test separately. 
For each genetic marker GWATCH calculates P-value of the association test and QAS. 

{\bf 2: The binary phenotype}\\
Currently, GWATCH uses the $\chi^2$ test for {\it codominant} alternative, and the combined two sided  $\chi^2$ and Fisher's exact test by default in other cases. Let $n_{\cdot j}=\sum_i n_{ij}$, $n_{i\cdot}=\sum_j n_{ij}$ and $n=n_{\bullet}=\sum_{ij} n_{ij}$.  For the combined test: the $\chi^2$ test is used under $n_{i\cdot}n_{\cdot j}\geq 5$ for all $i,j\in \{1,2\}$ and Fisher's exact test is used otherwise. The QAS is the Fisher's $z$-transformation of Pearson's correlation coefficient 
$
z'=\log((1+\rho)/(1-\rho))/2,
$
where 
$
\rho=\frac{n_{12}+2n_{13}-n_{2\cdot} t /n}{\sqrt{n_{2\cdot}(1-n_{2\cdot}/n)(n_{\cdot 2}+4n_{\cdot 3}-t^2/n)}}
$
and $t=n_{\cdot 2}+2n_{\cdot 3}$, for {\it codominant} test, and the log odds ratio $lor=\log(n_{11}n_{22}/(n_{21}n_{12}))$ for other tests.   

{\bf 3: The multilevel phenotype}\\
Currently, GWATCH uses the $\chi^2$ test under multilevel phenotype, even in the case of $2\times 2$ table after removing missed data. The P-value is calculated from the obtained contigency table $d'\times l'$ ($d',l'>1$; $d'\leq d$; $l'\leq 3$ under the codominant alternative and $l'\leq 2$ otherwise)
The QAS is the Fisher's $z$-transformation of Pearson's correlation coefficient 
$
z'=\log((1+\rho)/(1-\rho))/2,
$
where $\rho$ is the Pearson's correlation coefficient: 
$$
\rho=
\cases{
\frac{\sum_{i} (n_{i2}+2n_{i3})(i-1)-st /n}{\sqrt{\sum_i n_{i\cdot}(i-1)^2 -s^2/n)(n_{\cdot 2}+4n_{\cdot 3}-t^2/n)}},
\quad\mbox{under codominant alternative},\cr
\frac{\sum_{i} n_{i2}(i-1)-st /n}{\sqrt{\bigl(\sum_i n_{i\cdot}(i-1)^2 -s^2/n\bigr)n_{\cdot 2}(1-n_{\cdot 2}/n)}},
\quad\;\;\;\mbox{otherwise},
\cr
}
$$ 
where $s=\sum_i n_{i\cdot}(i-1)$ and $t=n_{\cdot 2}+2n_{\cdot 3}$. 


\end{document}

We consider a conditional model based on a distribution of random vector $(Y,X)$ with binary components, parametrized by $p_{j|i}=P(Y=j|X=i)$, $i,j=1,2$. Remark that 
$$
n_{11}-n_{1\cdot}n_{\cdot 1}/n=n_{12}-n_{1\cdot}n_{\cdot 2}/n=
n_{21}-n_{2\cdot}n_{\cdot 1}/n=n_{22}-n_{2\cdot}n_{\cdot 2}/n=x
$$
Therefore,
$$
X^2=x^2((n/n_{1\cdot}n_{\cdot 1})+(n/n_{2\cdot}n_{\cdot 1})+
(n/n_{1\cdot}n_{\cdot 2})+(n/n_{2\cdot}n_{\cdot 2})){\scriptsize =x^2 n^3/(n_{1\cdot} n_{2\cdot} n_{\cdot 1} n_{\cdot 2})}.
$$
Denote $n_{i}=n_{\cdot i}$ is the fixed mumber of individuals in $i$-th group, $i=1,2$.
Then, 
$$
X^2=n x^2/(\hat p_{1\cdot}\hat p_{2\cdot} n_1 n_2)=
n (n_{11} n_2 - n_{21} n_1)^2/(\hat p_{1\cdot}(1-\hat p_{1\cdot}) n_1 n_2).
$$
On the other hand, 
$$
\sqrt{n_i}(n_{1i}/n_i-p_{1|i})\Rightarrow N(0,p_{1|i}(1-p_{1|i})), \quad i=1,2 
$$
and $n_{11},n_{12}$ are independent random variables. Under null hypothesis   
$$
H_{0}:p_{1|1}=p_{1|2}=p_{1\cdot}
$$
and, therefore,
$$
X=\frac{n_2 n_{11}- n_1 n_{12}}{\sqrt{n n_1 n_2 p_{1\cdot} (1-p_{1\cdot})}}\Rightarrow N(0,1)
$$
as $n_1\to\infty$ and $n_2\to\infty$. Note that,
$$
\frac{n_{\cdot 1}}{n}\sim\frac{n_1}{n} p_{1|1}+\frac{n_2}{n} p_{1|2}=p_{1|1}-\frac{n_2}{n} (p_{1|1}-p_{1|2})=p_{1|1}-\frac{n_2}{n} \Delta_1
$$
and
$$
\frac{n_{\cdot 2}}{n}\sim1-\Bigl(\frac{n_{1}}{n} p_{1|1}+\frac{n_2}{n} p_{1|2}=1-p_{1|1}+\frac{n_2}{n} (p_{1|1}-p_{1|2})=1-p_{1|1}+\frac{n_2}{n} \Delta_1
$$
Then,
$$
\frac{n_2 n_{11}-n_1 n_{12}}{\sqrt{n_1 n_2 n_{1\cdot} n_{2\cdot}}/n}\sim
\frac{\sqrt{n_1 n_2} (p_{1|1}-p_{1|2})}{n\sqrt{(p_{1|1}-\frac{n_2}{n}\Delta_1)(1-p_{1|1}+\frac{n_2}{n}\Delta_1)}}
$$
and the non-centrality parameter is equal to 
$$
\mu_n^2\sim \frac{n_1 n_2}{n} \frac{\Delta_1^2}{(p_{1|1}-\frac{n_2}{n}\Delta_1)(1-p_{1|1}+\frac{n_2}{n}\Delta_1)}=
\frac{n_1 n_2}{n} \frac{(p_{1|1}-p_{2|1})^2}{(\frac{n_1}{n}p_{1|1}+\frac{n_2}{n}p_{1|2})(\frac{n_1}{n}p_{2|1}+\frac{n_2}{n}p_{2|2})}.
$$
 





 





\begin{center}
{\Large\bf Appendix} 
\end{center}

\begin{center}
{\large\bf Correlations}
\end{center}

{\bf Common multinomial approach.}

Let $\vec n=(n_{111},n_{112},\ldots,n_{222})^{T}$ and $\vec p=(p_{111},p_{112},\ldots,p_{222})^{T}$. It is well known that $\vec n$ has multinomial distribution and 
$$
\sqrt n(\vec n/n-\vec p)\Rightarrow \vec\xi
$$
and the vector $\vec \xi=(\xi_{111},\xi_{112},\ldots,\xi_{222})^{T}$ has the normal distribution $N(0,\Sigma)$ with 
$$
\var(\xi_{ijk})=p_{ijk}(1-p_{ijk})\;\;\mbox{and}\;\;
\cov(\xi_{ijk},\xi_{i'j'k'})=-p_{ijk}p_{i'j'k'}.
$$

Consider
$$
\tilde n=(n_{1\cdot 1},n_{1\cdot 2},n_{2\cdot 1},n_{2\cdot 2},n_{\cdot 11},n_{\cdot 12},n_{\cdot 21},n_{\cdot 22})^{T}
$$ 
and 
$$
\tilde p=(p_{1\cdot 1},p_{1\cdot 2},p_{2\cdot 1},p_{2\cdot 2},p_{\cdot 11},p_{\cdot 12},p_{\cdot 21},p_{\cdot 22})^{T},
$$ 
where $"\cdot"$ means the sum in corresponding index. Then,
$$
\sqrt n(\tilde n/n-\tilde p)\Rightarrow \vec\eta
$$
and $\vec\eta=(\eta_{1\cdot 1},\eta_{1\cdot 2},\eta_{2\cdot 1},\eta_{2\cdot 2},\eta_{\cdot 11},\eta_{\cdot 12},\eta_{\cdot 21},\eta_{\cdot 22})^{T}$ has mean zero normal distribution with the covariance matrix $\tilde\Sigma$ such that 
$$
\var(\eta_{i\cdot j})=p_{i\cdot j}(1-p_{i\cdot j}),
$$
$$
\var(\eta_{\cdot ij})=p_{\cdot ij}(1-p_{\cdot ij}),
$$
$$
\cov(\eta_{i\cdot j},\eta_{i'\cdot j'})=-p_{i\cdot j},p_{i'\cdot j'}\;\;\mbox{for}\;\; (i,j)\not =(i',j');
$$
$$
\cov(\eta_{\cdot ij},\eta_{\cdot i'j'})=-p_{\cdot ij},p_{\cdot i'j'}\;\;\mbox{for}\;\; (i,j)\not =(i',j');
$$
$$
\cov(\eta_{i\cdot j},\eta_{\cdot i'j'})=p_{ii'j}\indi_{\{j=j'\}}-p_{i\cdot j},p_{\cdot i'j'};
$$
or $\tilde\Sigma=$
\arraycolsep=0.3pt
$$
\scriptsize
=\!\!\left(\begin{array}{cccccccc}
p_{1\cdot 1}(1\!-\!p_{1\cdot 1}) & -p_{1\cdot 1}p_{1\cdot 2} & -p_{1\cdot 1}p_{2\cdot 1} & -p_{1\cdot 1}p_{2\cdot 2} & p_{111}\!-\!p_{1\cdot 1}p_{\cdot 11} & -p_{1\cdot 1}p_{\cdot 12} & p_{121}\!-\!p_{1\cdot 1}p_{\cdot 21} & -p_{1\cdot 1}p_{\cdot 22} \\
-p_{1\cdot 1}p_{1\cdot 2} & p_{1\cdot 2}(1\!-\!p_{1\cdot 2}) & -p_{1\cdot 2}p_{2\cdot 1} & -p_{1\cdot 2}p_{2\cdot 2} & -p_{1\cdot 2}p_{\cdot 11} & p_{112}\!-\!p_{1\cdot 2}p_{\cdot 12} &   -p_{1\cdot 2}p_{\cdot 21} & p_{122}\!-\!p_{1\cdot 2}p_{\cdot 22} \\
-p_{1\cdot 1}p_{2\cdot 1} & -p_{1\cdot 2}p_{2\cdot 1} & p_{2\cdot 1}(1\!-\!p_{2\cdot 1}) &  -p_{2\cdot 1}p_{2\cdot 2} & p_{211}\!-\!p_{2\cdot 1}p_{\cdot 11} & -p_{2\cdot 1}p_{\cdot 12} & p_{221}\!-\!p_{2\cdot 1}p_{\cdot 21} & -p_{2\cdot 1}p_{\cdot 22} \\
-p_{1\cdot 1}p_{2\cdot 2} & -p_{1\cdot 2}p_{2\cdot 2} & -p_{2\cdot 1}p_{2\cdot 2}  & p_{2\cdot 2}(1\!-\! p_{2\cdot 2}) & -p_{2\cdot 2}p_{\cdot 11} & p_{212}-p_{2\cdot 2}p_{\cdot 12} & -p_{2\cdot 2}p_{\cdot 21} & p_{222}-p_{2\cdot 2}p_{\cdot 22} \\
p_{111}\!-\!p_{1\cdot 1}p_{\cdot 11} & -p_{1\cdot 2}p_{\cdot 11} & p_{211}\!-\!p_{2\cdot 1}p_{\cdot 11} & -p_{2\cdot 2}p_{\cdot 11} & p_{\cdot 11}(1\!-\! p_{\cdot 11}) & -p_{\cdot 11} p_{\cdot 12} & -p_{\cdot 11} p_{\cdot 21} & -p_{\cdot 11} p_{\cdot 22} \\
-p_{1\cdot 1}p_{\cdot 12} & p_{112}\!-\!p_{1\cdot 2}p_{\cdot 12} & -p_{2\cdot 1}p_{\cdot 12} & p_{212}-p_{2\cdot 2}p_{\cdot 12} & -p_{\cdot 11} p_{\cdot 12} & p_{\cdot 12}(1\!-\! p_{\cdot 12})  & -p_{\cdot 12} p_{\cdot 21} & -p_{\cdot 12} p_{\cdot 22} \\
p_{121}\!-\!p_{1\cdot 1}p_{\cdot 21} & -p_{1\cdot 2}p_{\cdot 21} & p_{221}\!-\!p_{2\cdot 1}p_{\cdot 21} & -p_{2\cdot 2}p_{\cdot 21} & -p_{\cdot 11} p_{\cdot 21} & -p_{\cdot 12} p_{\cdot 21} & p_{\cdot 21}(1\!-\!p_{\cdot 21})  & -p_{\cdot 21} p_{\cdot 22} \\
-p_{1\cdot 1}p_{\cdot 22} & p_{122}\!-\!p_{1\cdot 2}p_{\cdot 22} & -p_{2\cdot 1}p_{\cdot 22} & p_{222}-p_{2\cdot 2}p_{\cdot 22} & -p_{\cdot 11} p_{\cdot 22} & -p_{\cdot 12} p_{\cdot 22} & -p_{\cdot 21} p_{\cdot 22}  &  p_{\cdot 22}(1\!-\!p_{\cdot 22}) 
\end{array}\right)\!\!.
$$

Using Teylor's decomposition 
$$
\log (x+\delta)=\log x + x^{-1} \delta + o(\delta)
$$
by delta method we obtain that 
$$
\sqrt n(\log (\tilde n)-\log(\tilde p))=A \sqrt n (\tilde n -\tilde p)+o_P(1)=
A \eta + o_P(1)=\vec\zeta,
$$
where $A=\diag(1/\tilde p)$ -- is the diagonal matrix with the elements of the vector $\tilde p$ on the diagonal. Then, the vector $\vec\zeta=(\zeta_{1\cdot 1},\zeta_{1\cdot 2},\zeta_{2\cdot 1},\zeta_{2\cdot 2},\zeta_{\cdot 11},\zeta_{\cdot 12},\zeta_{\cdot 21},\zeta_{\cdot 22})^{T}$ is normally distributed with the covariacnce matrix $\Upsilon=A\tilde \Sigma A$ with the elements
$$
\var(\zeta_{i\cdot j})=(1-p_{i\cdot j})/p_{i\cdot j},
$$
$$
\var(\eta_{\cdot ij})=(1-p_{\cdot ij})/p_{\cdot ij},
$$
$$
\cov(\eta_{i\cdot j},\eta_{i'\cdot j'})=\cov(\eta_{\cdot ij},\eta_{\cdot i'j'})=-1\;\;\mbox{for}\;\; (i,j)\not =(i',j');
$$
$$
\cov(\eta_{i\cdot j},\eta_{\cdot i'j'})=\frac{p_{ii'j}}{p_{i\cdot j},p_{\cdot i'j'}}\indi_{\{j=j'\}}-1;
$$
or $\Upsilon=$
\arraycolsep=0.3pt
$$
%\scriptsize
=\!\!\left(\begin{array}{cccccccc}
\frac{1\!-\!p_{1\cdot 1}}{p_{1\cdot 1}} & -1 & -1 & -1 & \frac{p_{111}}{p_{1\cdot 1}p_{\cdot 11}}-1 & -1 & \frac{p_{121}}{p_{1\cdot 1}p_{\cdot 21}}-1 & -1 \\
-1 & \frac{1\!-\!p_{1\cdot 2}}{p_{1\cdot 2}} & -1 & -1 & -1 & \frac{p_{112}}{p_{1\cdot 2}p_{\cdot 12}}-1 &   -1 & \frac{p_{122}}{p_{1\cdot 2}p_{\cdot 22}}-1 \\
-1 & -1 & \frac{1-p_{2\cdot 1}}{p_{2\cdot 1}} &  -1 & \frac{p_{211}}{p_{2\cdot 1}p_{\cdot 11}}-1 & -1 & \frac{p_{221}}{p_{2\cdot 1}p_{\cdot 21}}-1 & -1 \\
-1 & -1 & -1 & \frac{1-p_{2\cdot 2}}{p_{2\cdot 2}} & -1 & \frac{p_{212}}{p_{2\cdot 2}p_{\cdot 12}}-1 & -1 & \frac{p_{222}}{p_{2\cdot 2}p_{\cdot 22}}-1 \\
\frac{p_{111}}{p_{1\cdot 1}p_{\cdot 11}}-1 & -1 & \frac{p_{211}}{p_{2\cdot 1}p_{\cdot 11}}-1 & -1 & \frac{1\!-\! p_{\cdot 11}}{p_{\cdot 11}} & -1 & -1 & -1 \\
-1 & \frac{p_{112}}{\!p_{1\cdot 2}p_{\cdot 12}}-1 & -1 & \frac{p_{212}}{p_{2\cdot 2}p_{\cdot 12}}-1 & -1 & \frac{1-p_{\cdot 12}}{p_{\cdot 12}}  & -1 & -1 \\
\frac{p_{121}}{p_{1\cdot 1}p_{\cdot 21}}-1 & -1 & \frac{p_{221}}{p_{2\cdot 1}p_{\cdot 21}}-1 & -1 & -1 & -1 & \frac{1\!-\!p_{\cdot 21}}{p_{\cdot 21}}  & -1 \\
-1 & \frac{p_{122}}{p_{1\cdot 2}p_{\cdot 22}}-1 & -1 & \frac{p_{222}}{p_{2\cdot 2}p_{\cdot 22}}-1 & -1 & -1 & -1  &  \frac{1\!-\!p_{\cdot 22}}{p_{\cdot 22}}
\end{array}\right)\!\!=
$$
$$
=\!\left(\begin{array}{cccccccc}
p_{1\cdot 1}^{-1} & 0 & 0 & 0 & \frac{p_{111}}{p_{1\cdot 1}p_{\cdot 11}}  & 0 & \frac{p_{121}}{p_{1\cdot 1}p_{\cdot 21}}  & 0 \\
0 & p_{1\cdot 2}^{-1} & 0 & 0 & 0 & \frac{p_{112}}{p_{1\cdot 2}p_{\cdot 12}}  & 0 & \frac{p_{122}}{p_{1\cdot 2}p_{\cdot 22}} \\
0 & 0 & p_{2\cdot 1}^{-1} &  0 & \frac{p_{211}}{p_{2\cdot 1}p_{\cdot 11}}  & 0 & \frac{p_{221}}{p_{2\cdot 1}p_{\cdot 21}}  & 0 \\
0 & 0 & 0 & p_{2\cdot 2}^{-1} & 0 & \frac{p_{212}}{p_{2\cdot 2}p_{\cdot 12}}  & 0 & \frac{p_{222}}{p_{2\cdot 2}p_{\cdot 22}}  \\
\frac{p_{111}}{p_{1\cdot 1}p_{\cdot 11}}  & 0 & \frac{p_{211}}{p_{2\cdot 1}p_{\cdot 11}}  & 0 & p_{\cdot 11}^{-1} & 0 & 0 & 0 \\
0 & \frac{p_{112}}{\!p_{1\cdot 2}p_{\cdot 12}} & 0 & \frac{p_{212}}{p_{2\cdot 2}p_{\cdot 12}}  & 0 & p_{\cdot 12}^{-1}  & 0 & 0 \\
\frac{p_{121}}{p_{1\cdot 1}p_{\cdot 21}}  & 0 & \frac{p_{221}}{p_{2\cdot 1}p_{\cdot 21}}  & 0 & 0 & 0 & p_{\cdot 21}^{-1}  & 0 \\
0 & \frac{p_{122}}{p_{1\cdot 2}p_{\cdot 22}}  & 0 & \frac{p_{222}}{p_{2\cdot 2}p_{\cdot 22}}  & 0 & 0 & 0  &  p_{\cdot 22}^{-1} 
\end{array}\right)\!\!-\mbox{\bf 1},
$$
where {\bf 1} is the matrix of units.

Let $T=(T_1,T_2)^{T}$, where 
$$
T_1=\log(n_{1\cdot 1}n_{2\cdot 2}/(n_{1\cdot 2} n_{2\cdot 1}))=
\log n_{1\cdot 1}-\log n_{1\cdot 2} - \log n_{2\cdot 1} + \log n_{2\cdot 2}
$$
and 
$$
T_2=\log(n_{\cdot 11}n_{\cdot 22}/(n_{\cdot 12} n_{\cdot 21}))=
\log n_{\cdot 11}-\log n_{\cdot 12} - \log n_{\cdot 21} + \log n_{\cdot 22}
$$
Then 
$$
\sqrt n (T-T_0)\Rightarrow \nu
$$
and $\nu=B \zeta$, where
\arraycolsep=10pt
$$
B=\left(
\begin{array}{cccccccc}
1 & -1 & -1 & 1 & 0 & 0 & 0 & 0 \\
0 &  0 &  0 & 0 & 1 & -1 & -1 & 1 \\ 
\end{array}
\right)
$$
Taking into account that $B${\bf 1}$B^{T}=0$ we obtain that $\nu\sim N(0,\Gamma)$
and 
$$
\var(T_1)=\sum_{ij} p_{i\cdot j}^{-1}, \quad \var(T_2)=\sum_{ij} p_{\cdot ij}^{-1},
$$
$$
\cov(T_1,T_2)=\sum_{ijk}(-1)^{|(i,j)|} \frac{p_{ijk}}{p_{i\cdot k}p_{\cdot jk}},
$$
where $|(i,j)|=\indi_{\{i\not =j\}}$. In explicit form 
\arraycolsep=5pt
\renewcommand{\arraystretch}{2}
$$
\Gamma=\left(
\begin{array}{cc}
\frac{1}{p_{1\cdot 1}}+\frac{1}{p_{1\cdot 2}} +\frac{1}{p_{2\cdot 1}} + \frac{1}{p_{2\cdot 2}} & 
\mbox{
\begin{minipage}{168pt}
$\frac{p_{111}}{p_{1\cdot 1}p_{\cdot 11}}-\frac{p_{121}}{p_{1\cdot 1}p_{\cdot 12}} +
\frac{p_{112}}{p_{1\cdot 2}p_{\cdot 12}} - \frac{p_{122}}{p_{1\cdot 2}p_{\cdot 22}}+
\vspace{+1mm}
\\
+\frac{p_{221}}{p_{2\cdot 1}p_{\cdot 21}}-\frac{p_{211}}{p_{2\cdot 1}p_{\cdot 11}} +
\frac{p_{222}}{p_{2\cdot 2}p_{\cdot 22}} - \frac{p_{212}}{p_{2\cdot 2}p_{\cdot 12}}$
\end{minipage} 
}
\\
\mbox{
\begin{minipage}{168pt}
$\frac{p_{111}}{p_{1\cdot 1}p_{\cdot 11}}-\frac{p_{121}}{p_{1\cdot 1}p_{\cdot 12}} +
\frac{p_{112}}{p_{1\cdot 2}p_{\cdot 12}} - \frac{p_{122}}{p_{1\cdot 2}p_{\cdot 22}}+
\vspace{+1mm}
\\
+\frac{p_{221}}{p_{2\cdot 1}p_{\cdot 21}}-\frac{p_{211}}{p_{2\cdot 1}p_{\cdot 11}} +
\frac{p_{222}}{p_{2\cdot 2}p_{\cdot 22}} - \frac{p_{212}}{p_{2\cdot 2}p_{\cdot 12}}$
\end{minipage} 
} 
&  \frac{1}{p_{\cdot 11}}+\frac{1}{p_{\cdot 12}} +\frac{1}{p_{\cdot 21}} + \frac{1}{p_{\cdot 22}}
\end{array}
\right).
$$

{\bf Conditional approach.} In multiple testing model multiple SNP are calculated for any fixed individual, and therefore, it is reasonable to use conditional model. Consider two SNP conditional model. The parameters $p_{ij|k}$ are probabilities of different SNP combinations $(i,j)$ under $k=0$ (no disease) and under $k=1$ (disease). Let $n_1$ and $n_2$ be the numbes of non infected and infected patients respectively, $n=n_1+n_2$. The conditional multinomial distributions under given $k$ are pecified by the following tables under $k=1,2$:
\arraycolsep=3pt
$$
%\scriptsize
=\left(\begin{array}{cccc} 
p_{11|k}(1-p_{11|k}) & -p_{11|k} p_{12|k} & -p_{11|k} p_{21|k} & -p_{11|k} p_{22|k} \\
-p_{11|k} p_{12|k} & p_{12|k}(1-p_{12|k}) & -p_{12|k} p_{21|k} & -p_{12|k} p_{22|k} \\
-p_{11|k} p_{21|k} & -p_{12|k} p_{21|k} & p_{21|k}(1-p_{21|k}) & -p_{21|k} p_{22|k} \\
-p_{11|k} p_{22|k} & -p_{12|k} p_{22|k} & -p_{21|k} p_{22|k} & p_{22|k}(1-p_{22|k}) 
\end{array}
\right)
$$
and $\{n_{ij1},i,j=1,2\}$ are conditionally independent with $\{n_{ij2},i,j=1,2\}$.
The conditional distributions in contingency tables $\tilde n_k=(n_{1\cdot k},n_{2\cdot k},n_{\cdot 1k},n_{\cdot 2k})$ we have the limit correlation matrix for $\sqrt{n_k}(\tilde n_k/n_k -\tilde p_{|k})$:
\arraycolsep=5pt
$$
%\scriptsize
\left(\begin{array}{cccc} 
p_{1\cdot|k}(1-p_{1\cdot|k}) & -p_{1\cdot|k} p_{2\cdot |k} & p_{11|k}-p_{1\cdot|k} p_{\cdot 1|k} & p_{12|k}-p_{1\cdot|k} p_{\cdot 2|k} \\
-p_{1\cdot|k} p_{2\cdot|k} & p_{2\cdot|k}(1-p_{2\cdot|k}) & p_{21|k}-p_{2\cdot|k} p_{\cdot 1|k} & p_{22|k}-p_{2\cdot|k} p_{\cdot 2|k} \\
p_{11|k}-p_{1\cdot|k} p_{\cdot 1|k} & p_{21|k}-p_{\cdot 1|k} p_{2\cdot|k} & p_{\cdot 1|k}(1-p_{\cdot 1|k}) & -p_{\cdot 1|k} p_{\cdot 2|k} \\
p_{12|k}-p_{1\cdot|k} p_{\cdot 2|k} & p_{22|k}-p_{2\cdot|k} p_{\cdot 2|k} & -p_{\cdot 1|k} p_{\cdot 2|k} & p_{\cdot 2|k}(1-p_{\cdot 2|k}) 
\end{array}
\right)
$$
and by delta method we obtain limit covariance matrix for $\sqrt{n_k}(\log\tilde n_k/n_k -\log\tilde p_{|k})$:
$$
%\scriptsize
\left(\begin{array}{cccc} 
\frac{1}{p_{1\cdot|k}} & 0 & \frac{p_{11|k}}{p_{1\cdot|k} p_{\cdot 1|k}} & \frac{p_{12|k}}{p_{1\cdot|k} p_{\cdot 2|k}} \\
0 & \frac{1}{p_{2\cdot|k}} & \frac{p_{21|k}}{p_{2\cdot|k} p_{\cdot 1|k}} & \frac{p_{22|k}}{p_{2\cdot|k} p_{\cdot 2|k}} \\
\frac{p_{11|k}}{p_{1\cdot|k} p_{\cdot 1|k}} & \frac{p_{21|k}}{p_{\cdot 1|k} p_{2\cdot|k}} & \frac{1}{p_{\cdot 1|k}} & 0 \\
\frac{p_{12|k}}{p_{1\cdot|k} p_{\cdot 2|k}} & \frac{p_{22|k}}{p_{2\cdot|k} p_{\cdot 2|k}} & 0 & \frac{1}{p_{\cdot 2|k}}
\end{array}
\right)-\mbox{\bf 1}.
$$
Set $r_k=n/n_k$. Containing these objects into $T=(T_1,T_2)$ we obtain that the limit Gaussian distribution for $\sqrt n(T-T_0)$ have the following correlation matrix
\arraycolsep=5pt
\renewcommand{\arraystretch}{2}
$$
\Gamma=\left(
\begin{array}{cc}
\frac{r_1}{p_{1\cdot|1}}+\frac{r_2}{p_{1\cdot|2}} +\frac{r_1}{p_{2\cdot 1}} + \frac{r_2}{p_{2\cdot 2}} & 
\mbox{
\begin{minipage}{178pt}
$\frac{r_1 p_{11|1}}{p_{1\cdot |1}p_{\cdot 1|1}}\!-\!\frac{r_1 p_{12|1}}{p_{1\cdot |1}p_{\cdot 2|2}}\! +\!
\frac{r_2 p_{11|2}}{p_{1\cdot |2}p_{\cdot 1|2}}\! -\! \frac{r_2 p_{12|2}}{p_{1\cdot |2}p_{\cdot 2|2}}\!+
\vspace{+1mm}
\\
+\!\frac{r_1 p_{22|1}}{p_{2\cdot |1}p_{\cdot 2|1}}\!-\!
\frac{r_1 p_{21|1}}{p_{2\cdot |1}p_{\cdot 1|1}}\!+\!
\frac{r_2 p_{22|2}}{p_{2\cdot |2}p_{\cdot 2|2}}\!-\! \frac{r_2 p_{21|2}}{r_2 p_{2\cdot |2}p_{\cdot 1|2}}$
\end{minipage} 
}
\\
\mbox{
\begin{minipage}{178pt}
$\frac{r_1 p_{11|1}}{p_{1\cdot |1}p_{\cdot 1|1}}\!-\!\frac{r_1 p_{12|1}}{p_{1\cdot |1}p_{\cdot 2|2}}\! +\!
\frac{r_2 p_{11|2}}{p_{1\cdot |2}p_{\cdot 1|2}}\! -\! \frac{r_2 p_{12|2}}{p_{1\cdot |2}p_{\cdot 2|2}}\!+
\vspace{+1mm}
\\
+\!\frac{r_1 p_{22|1}}{p_{2\cdot |1}p_{\cdot 2|1}}\!-\!\frac{r_1 p_{21|1}}{p_{2\cdot |1}p_{\cdot 1|1}}\!+\!
\frac{r_2 p_{22|2}}{p_{2\cdot |2}p_{\cdot 2|2}}\!-\!\frac{r_2 p_{21|2}}{p_{2\cdot |2}p_{\cdot 1|2}}$
\end{minipage} 
} 
&  \frac{r_1}{p_{\cdot 1|1}}+\frac{r_2}{p_{\cdot 1|2}} +\frac{r_1}{p_{\cdot 2|1}} + \frac{r_2}{p_{\cdot 2|2}}
\end{array}
\right).
$$

\noindent
{\bf Chi-square tests.} The $\chi^2$ test statistics for independence based on two contingency tables 
$
X^2_k=\sum_{i=1}^2 \sum_{j=1}^2 y_{ijk}^2
$, where 
$$
y_{ij1}=(n_{\cdot ij}-\hat n_{\cdot ij})/\sqrt{\hat n_{\cdot ij}} \quad \mbox{and}\quad
y_{ij2}=(n_{i\cdot j}-\hat n_{i\cdot j})/\sqrt{\hat n_{i\cdot j}},
$$
$$
\hat n_{\cdot ij}= n_{\cdot\cdot j} n_{\cdot i\cdot}/ n,
\quad
\hat n_{i\cdot j}= n_{i \cdot\cdot} n_{\cdot\cdot j}/ n,
$$
and $n=\sum_{ijk} n_{ijk}$ is the total number of observations.

It is well known that the asymptotic distributions of $X^2_k$ are $\chi^2_1$; 
the asymptotic distribution of any single statistic is represented as square of some standard normal distribution and the common asymptotic distribution of $(X^2_1,X^2_2)$ can be represented in terms of bivariate normal distribution. 
Consider 
$
\tilde x=(\tilde n - n\tilde p) / \sqrt{n\tilde p}
$
(where the operation "/" were used component-wise). The random vector 
$\tilde x$ \ is \ asymptotically \ normal \
$
\tilde x \Rightarrow \vec\kappa
$ \
and \ $\vec\kappa=(\kappa_{1\cdot 1},\kappa_{1\cdot 2},\kappa_{2\cdot 1},\kappa_{2\cdot 2},\kappa_{\cdot 11},\kappa_{\cdot 12},\kappa_{\cdot 21},\kappa_{\cdot 22})^{T}$ has mean zero multivariate normal distribution with 
$$
\var(\kappa_{i\cdot j})=1-p_{i\cdot j},
$$
$$
\var(\kappa_{\cdot ij})=1-p_{\cdot ij},
$$
$$
\cov(\kappa_{i\cdot j},\kappa_{i'\cdot j'})=-\sqrt{p_{i\cdot j},p_{i'\cdot j'}}\;\;\mbox{for}\;\; (i,j)\not =(i',j');
$$
$$
\cov(\kappa_{\cdot ij},\kappa_{\cdot i'j'})=-\sqrt{p_{\cdot ij},p_{\cdot i'j'}}\;\;\mbox{for}\;\; (i,j)\not =(i',j');
$$
$$
\cov(\kappa_{i\cdot j},\kappa_{\cdot i'j'})=
\frac{p_{ii'j}}{\sqrt{p_{i\cdot j}p_{\cdot i'j}}}\indi_{\{j=j'\}}-\sqrt{p_{i\cdot j},p_{\cdot i'j'}}.
$$
In other words, the covariance matrix of $\vec \kappa$ is  $\Sigma^{\circ}=P\tilde{\Sigma} P$, where $P$ is the diagonal matrix with $(1/\sqrt{p_{1\cdot 1}},1/\sqrt{p_{1\cdot 2}},1/\sqrt{p_{2\cdot 1}},1/\sqrt{p_{2\cdot 2}},1/\sqrt{p_{\cdot 11}},1/\sqrt{p_{\cdot 12}},1/\sqrt{p_{\cdot 21}},1/\sqrt{p_{\cdot 22}})$ at the diagonal. Let 
\renewcommand{\arraystretch}{0.9}
$$
\Sigma^{\circ}=
\left(\begin{array}{cc}
\Sigma^{\circ}_{11}  & \Sigma^{\circ}_{12}\\ 
\Sigma^{\circ}_{21}  & \Sigma^{\circ}_{22}
\end{array}
\right)
$$
where $\Sigma^{\circ}_{ii}$ is the covariance matrix of $\vec\kappa_i$, $i=1,2$.
Remark that $\tilde y_i=A_i \tilde x_i+\tilde w_i$, with $\tilde w_i\to_P 0$ as $n\to\infty$,
$
A_i=I-B_i(B^{T}_iB_i)^{-1}B^{T}_i
$,  
$B_i=P_i J_i$ ($P_j$ is the matrix of corresponding minor of $P$)  and $J_i$ is the Jacobi matrix of the parameterization in $i$-th table under null hypothesis, $i=1,2$ (see Cramer (1946), 30.3). Under the parameterizations by $u_1=p_{1\cdot \cdot}$, $v_1=p_{\cdot\cdot 1}$
and $u_2=p_{\cdot 1 \cdot}$, $v_2=p_{\cdot\cdot 1}$,
\renewcommand{\arraystretch}{1}
$$
J_i=\left(\begin{array}{cccc}
v_i  & 1-v_i & -v_i & -(1-v_i) \\ 
u_i  & -u_i   & 1-u_i& -(1-u_i) 
\end{array}
\right)^{T}.
$$
Then, the matrix $C_i=A_i\Sigma_{ii}^{\circ} A^{T}_i$ is idempotent of rank $1$ and 
$X_i^2=z_i^2$, where $z_i=d_i^{T} y_i$ has the standard normal distribution with $d_i$ is the eigenvector corresponding to the non-zero eigenvalue of $C_i$. Then, 
$$
\rho=\cov(z_1,z_2)=d_1^{T}\Sigma_{12}^{\circ} d_2.
$$

{\bf Conditional approach.}
Remark that 
$$
n_{11}-n_{1\cdot}n_{\cdot 1}/n=n_{12}-n_{1\cdot}n_{\cdot 2}/n=
n_{21}-n_{2\cdot}n_{\cdot 1}/n=n_{22}-n_{2\cdot}n_{\cdot 2}/n=x
$$
Therefore,
$$
X^2=x^2((n/n_{1\cdot}n_{\cdot 1})+(n/n_{2\cdot}n_{\cdot 1})+
(n/n_{1\cdot}n_{\cdot 2})+(n/n_{2\cdot}n_{\cdot 2})){\scriptsize =x^2 n^3/(n_{1\cdot} n_{2\cdot} n_{\cdot 1} n_{\cdot 2})}.
$$
Denote $n_{i}=n_{\cdot i}$ is the fixed mumber of individuals in $i$-th group, $i=1,2$.
Then, 
$$
X^2=n x^2/(\hat p_{1\cdot}\hat p_{2\cdot} n_1 n_2)=
n (n_{11} n_2 - n_{21} n_1)^2/(\hat p_{1\cdot}(1-\hat p_{1\cdot}) n_1 n_2).
$$
On the other hand, 
$$
\sqrt{n_i}(n_{1i}/n_i-p_{1|i})\Rightarrow N(0,p_{1|i}(1-p_{1|i})), \quad i=1,2 
$$
and $n_{11},n_{12}$ are independent random variables. Under null hypothesis   
$$
H_{0}:p_{1|1}=p_{1|2}=p_{1\cdot}
$$
and, therefore,
$$
X=\frac{n_2 n_{11}- n_1 n_{12}}{\sqrt{n n_1 n_2 p_{1\cdot} (1-p_{1\cdot})}}\Rightarrow N(0,1)
$$
as $n_1\to\infty$ and $n_2\to\infty$. 

Now we consider two contingency tables and
$$
X_1=\frac{n_2 n_{1\cdot 1}- n_1 n_{1\cdot 2}}{\sqrt{n n_1 n_2 p_{1\cdot\cdot} (1-p_{1\cdot\cdot})}},
\quad
X_2=\frac{n_2 n_{\cdot 11}- n_1 n_{\cdot 12}}{\sqrt{n n_1 n_2 p_{\cdot 1\cdot} (1-p_{\cdot 1\cdot})}}
$$
are the corresponding markers and 
$$
(X_1,X_2)^{T}\Rightarrow N(0,\Sigma)),
$$
where 
\renewcommand{\arraystretch}{1}
$
\Sigma=
\left(
\begin{array}{cc}
1 & \rho \\
\rho & 1 \\
\end{array}
\right)
$ and 
$$
\rho=\cov(X_1,X_2)=\frac{n_2^2 \cov(n_{1\cdot 1},n_{\cdot 11})- n_1^2 \cov(n_{1\cdot 2},n_{\cdot 12})}{n n_1 n_2 \sqrt{p_{1\cdot\cdot} (1-p_{1\cdot\cdot})p_{\cdot 1\cdot} (1-p_{\cdot 1\cdot})}}=
$$
$$
=\frac{n_2^2 n_1 (p_{11|1}p_{22|1}-p_{21|1}p_{12|1}) + 
n_1^2 n_2 (p_{11|2}p_{22|2}-p_{21|2}p_{12|2})}{n n_1 n_2 \sqrt{p_{1\cdot\cdot} (1-p_{1\cdot\cdot})p_{\cdot 1\cdot} (1-p_{\cdot 1\cdot})}}=
$$
$$
=\frac{n_2 (p_{11|1}p_{22|1}-p_{21|1}p_{12|1}) + 
n_1 (p_{11|2}p_{22|2}-p_{21|2}p_{12|2})}{n \sqrt{p_{1\cdot\cdot} (1-p_{1\cdot\cdot})p_{\cdot 1\cdot} (1-p_{\cdot 1\cdot})}}.
$$



Under null hypothesis 

In case of conditional design 


\end{document}